%! Author = tobias
%! Date = 01.03.22

% Preamble
\documentclass[11pt]{article}
\title{Gedächtnisprotokoll Jagdprüfung}
\author{Tobias Famos}

% Packages
\usepackage{amsmath}

% Document
\begin{document}
    \maketitle

    \section{Gesetzeskunde}
    Alles wie erwartet.
    48 Fragen, bei 38 korrekt eine 6.

    \section{Wildtierbiliogie}
    \begin{enumerate}
        \item Welcher ist marder-artig? (Je ein bild von: Dachs, Eichhörnchen, Waschbär, Fuchs)
        \item Vogel erkennen ca 6 Fotos quer über alle Ordnungen (Bitte nennen Sie nur die Art)
        \item Bild von Hirsch: Welches Tier sehen sie hier und sprechen Sie es an?
        \item Bild von Goldschakal: Welches Tier sehen Sie hier?
        \item Zwei Hörner einer Gams  (von der gleichen Gams): Von welchem Tier stammen diese Hörner? Wie Alt? Welches Geschelcht?
        \item Horn eines Steinbocks: Wie Alt?
        \item Geweihzyklus vom Hirsch
        \item Länge Zyklus vom Hirsch
        \item Können Sie vom Geweih eines Hirsches auf sein Alter schliessen?
        \item Wie verlöuft die Brunft beim Hirsch?
        \item Welches Tier sehen sie auf dem Bild? (Auerhahn)
        \item Wie nennt man die Paarungszeit beim Auerwild?
        \item Wie verläuft die Balz vom Auerwild?
        \item Beschreiben Sie den Lebensraum vom Auerwild
    \end{enumerate}

    \section{Wild und Umwelt}
    \begin{enumerate}
        \item Grösse der Population von Gamswild im Frühjahr im Kanton Graubünden.
        \item Grösse der Population von Steinwild im Frühjahr im Kanton Graubünden.
        \item Waldzustände
        \item Bild von Fragmentierter Kulturlandschaft: Wie beinflusst eine solche Landschaft das Wild?
        \item Wer ist der grösste Feind der Biodiversität?
        \item Bild von Lebensraum: Beschreiben Sie den Lebensraum und welche Lebewesen vorkommen
        \item Abbildung aus Buch von Stoffkerislauf: Was sehen Sie hier und beschreiben Sie?
        \item Was ist Bioakkumulation?
    \end{enumerate}

    \section{Jagdkunde}
    \begin{enumerate}
        \item Pirschzeichen
        \item Was kann ein Jäger das Jahr über tun um seine treffsicherheit zu garantieren?
        \item Wie verhalten Sie sich unmittelbar (30 Sekunen) vor dem Schuss?
        \item Wie verhalten Sie sich unmittelbar nach dem Schuss (nächste 5 Minuten)?
        \item Bild von Hirschstier auf Kuppe: Schiessen sie hier?
        \item Bild von Hirschkuh auf Kuppe: Schiessen sie hier?
        \item Was ist ein Kammerstich?
        \item Wie nähern sie sich geschossenem Wild?
        \item Was tun Sie wenn das Wild nicht im Feuer liegt?
        \item Wie fordern Sie einen Schweisshund an? (Folgefrage: und müssen Sie den Wildhüter informieren?)
    \end{enumerate}
\section{Generelle Anmerkungen}
    \begin{itemize}
        \item Meine Prüfer halfen mir, wennn ich die Fragen nicht wusste, oder nicht wusste was der Prüfer von mir gerne hören würde
        \item Der Prüfer sagt einem nicht, ob die Antwort korrekt war oder nicht, aber es ist ziemlich klar an der Reaktion einiger Prüfer erkennbar.
        \item Es müssen alle Fragen durchgemacht werden.
        Der Prüfer kann einen nur anhand des vorhanden Fragebogens bewerten.
        Wir konnten nur die Hälfte der Fragen beantworten gilt nicht.
        Wenn früher fertig dann ist früher schluss, wenn zu spät fertig wird ein wenig überzogen und die restlichen Fragen werden im Schnelldurchlauf beantwortet.
        Nichts wenn ich halt mehr sage am Anfang kann der Prüfer weniger Fragen stellen.
    \end{itemize}


\end{document}